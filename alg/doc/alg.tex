\documentclass[11pt]{article}

\usepackage{amsmath}
\usepackage{amsfonts}
\usepackage{array}

\title{RAIM Co-processor Algorithm}
\author{Michael Grieco}
\date{}

\begin{document}

\section{Main definitions}

\begin{table}[htbp] \centering
  \caption{Constant definitions}\label{tab:exp}
  \begin{tabular}{|m{2cm}|m{11cm}|}
    \hline
    \textbf{Name} & \textbf{Description}\\ \hline
    $k$ & Subset index. The initial value, $k=0$ denotes the all-in-view subset assuming no faults.\\ \hline
    $N_{sat}$ & Integer number of satellites.\\ \hline
    $N_{const}$ & Integer number of constellations.\\ \hline
    $N_{ss}$ & Integer subsets for which to calculate solutions.\\ \hline
  \end{tabular}
\end{table}

\section{RAIM subset least-squares estimations}

\subsection{Definitions}

\begin{table}[htbp] \centering
  \caption{Definitions}
  \begin{tabular}{|m{2cm}|m{2cm}|m{2cm}|m{7cm}|}
    \hline
    \textbf{Notation} & \textbf{Relation} & \textbf{Dimensions} & \textbf{Description}\\ \hline
    $x_{SV_i}$ & $SV_i$ & $3 \times 1$ & Unit vector pointing from the linearized receiver position to the satellite position in ENU coordinates\\ \hline
    $c_{SV_i}$ & $SV_i$ & $1 \times 1$ & Constellation number.\\ \hline
    $\sigma_{tropo,i}^2$ & $SV_i$ & $1 \times 1$ & Variance for the tropospheric delay.\\ \hline
    $\sigma_{user,i}^2$ & $SV_i$ & $1 \times 1$ & Variance for the user delay based on line-of-sight geometry.\\ \hline
    $\sigma_{URA,i}^2$ & $SV_i$ & $1 \times 1$ & Variance of the clock and ephemeris error used for integrity.\\ \hline
    $\sigma_{URE,i}^2$ & $SV_i$ & $1 \times 1$ & Variance of the clock and ephemeris error used for accuracy and continuity.\\ \hline
    $b_{nom,i}$ & $SV_i$ & $1 \times 1$ & Maximum nominal bias for integrity.\\ \hline
    $idx_{ss,k}$ & $SS_k$ & $1 \times 1$ & $N_{sat}$-bit activation string to indicate satellites included in a subset.\\ \hline
    $\vec{y}$ & $SS_k$ & $N_{sat} \times 1$ & Vector to multiply with the weighted least-squares matrix.\\ \hline
  \end{tabular}
\end{table}

\subsection{Equations}

Initial matrices the RPU must calculate:

$C_{int}(i,i) \equiv \sigma_{URA,i}^2 + \sigma_{tropo,i}^2 + \sigma_{user,i}^2$

$C_{acc}(i,i) \equiv \sigma_{URE,i}^2 + \sigma_{tropo,i}^2 + \sigma_{user,i}^2$

$W \equiv C_{int}^{-1} \Rightarrow W(i,i)=\frac{1}{\sigma_{URA,i}^2 + \sigma_{tropo,i}^2 + \sigma_{user,i}^2}$

All-in-view iterative solution:

$\Delta \hat{x}_j=(G^T W G)^{-1} G^T W \Delta PR_j$

$\Delta PR_j(i)=PR_i-|\vec{x}_{SV_{i}}-\hat{x}_j|$

$\hat{x}_{j+1} = \hat{x}_j + \Delta \hat{x}_j$

$\vec{y} \equiv \Delta PR_J$

Least-squares matrix for each subset:

$S^{(k)} = (G^T W^{(k)} G)^{-1} G^T W^{(k)} \in \mathbb{R}^{(3+N_{const}) \times N_{sat}}$

$W^{(k)} \in \mathbb{R}^{N_{sat} \times N_{sat}}$

$G \in \mathbb{R}^{N_{sat} \times (3+N_{const})} \Rightarrow G^T \in \mathbb{R}^{(3+N_{const}) \times N_{sat}}$

$\Rightarrow G^T W^{(k)} \in \mathbb{R}^{(3+N_{const}) \times N_{sat}}$

$\Rightarrow G^T W^{(k)} G, (G^T W^{(k)} G)^{-1} \in \mathbb{R}^{(3+N_{const}) \times (3+N_{const})}$

$\Rightarrow S^{(k)} = (G^T W^{(k)} G)^{-1} G^T W^{(k)} = (W^{(k)1/2} G)^{\dagger} * W^{(k)1/2} \in \mathbb{R}^{(3+N_{const}) \times N_{sat}}$

The following equations are applicable for all subsets ($\forall \ k \in [1,N_{ss}]$) and the three ENU components ($\forall \ q \in \{1,2,3\}$). Subset solution separations, along with variances and biases:

$\Delta \hat{x}^{(k)} = \hat{x}^{(k)} - \hat{x}^{(0)} = (S^{(k)} - S^{(0)})\vec{y} \in \mathbb{R}^{(3+N_{const})}$

$\sigma_q^{(k)2} = [(G^T W^{(k)} G)^{-1}](q,q)$

$b_q^{(k)} = \sum_{i=1}^{N_{sat}}|S^{(k)}(q,i)|*b_{nom,i}$

$\sigma_{ss,q}^{(k)2} = [(S^{(k)}-S^{(0)})C_{acc}(S^{(k)}-S^{(0)})^T](q,q)$

\section{Fault detection and exclusion}

\subsection{Definitions}

\begin{table}[htbp] \centering
  \caption{Definitions}
  \begin{tabular}{|m{2cm}|m{2cm}|m{2cm}|m{7cm}|}
    \hline
    \textbf{Notation} & \textbf{Relation} & \textbf{Dimensions} & \textbf{Description}\\ \hline
    $K_{fa}$ & conf & $3 \times 1$ & Threshold factors derived from the continuity budgets allocated for the vertical and horizontal modes. See (19) and (20) in [X] for more details.\\ \hline
    $K_{fa,r}$ & conf & $1 \times 1$ & Threshold factors dervied from the continuity budgets allocated for the local test. See (22) and (23) in [X] for more details.\\ \hline
    $idx_{ss,k}$ & $SS_k$ & $1 \times 1$ & $N_{sat}$-bit activation string to indicate satellites included in a subset.\\ \hline
  \end{tabular}
\end{table}

\subsection{Calculations}

Requirements

Solution: $N_{sat} \ge 3 + N_{const}; N_{const} \ge 1$

Fault detection: $N_{sat} \ge 3 + N_{const} + 1; N_{const} \ge 1$

Fault exclusion: $N_{sat} \ge 3 + N_{const} + 2; N_{const} \ge 1$

Global test for each subset $k$:

$|\Delta \hat{x}^{(k)}| = |\hat{x}^{(k)} - \hat{x}^{(0)}| \le \sigma_{ss,q}^{(k)2} * K_{fa,q}$

$K_{fa,1} = K_{fa,2} = Q^{-1}(\frac{P_{FA\_HOR}}{4N_{fault modes}})$

$K_{fa,3} = Q^{-1}(\frac{P_{FA\_VERT}}{2N_{fault modes}})$

$Q^{-1}(p)$ is the $(1-p)$-quantile of a normal distribution.

Local test for each satellite vehicle $i$:

$|y_i| \le \sqrt{C_{int}(i,i)} * K_{fa,r}$

$K_{fa,r} = Q^{-1}(\frac{P_{FA}}{2})$

\end{document}
