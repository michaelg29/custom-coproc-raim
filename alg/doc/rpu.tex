\documentclass[11pt]{article}

\usepackage{amsmath}
\usepackage{amsfonts}
\usepackage{array}

\title{RAIM Co-processor Appendix}
\author{Michael Grieco}
\date{}

\begin{document}

\section{Registers}

\begin{table}[htbp] \begin{center}
  \caption{Internal RPU registers. All sizes are in bits. The notation explains what each indexed element corresponds to. Each register has external and internal access permissions, with "R" meaning readable, "W" meaning writeable, "RW" meaning readable and writeable, "-" meaning inaccessible, and "*" indicating a virtual register.}
  \label{tab:reg}
  \begin{tabular}{|m{1.5cm}|m{1.5cm}|m{6cm}|m{1cm}|m{1cm}|}
    \hline
    \textbf{Name} & \textbf{Size (bits)} & \textbf{Notation/Description} & \textbf{Ext.} & \textbf{Int.}\\ \hline

    $LOS$ & $3 \times 32$ & The current SV's 3-dimensional LOS vector. & W & *\\ \hline

    $C$ & $2$ & Constellation index for the current SV. The values start from $0$ and increment by $1$ for each additional constellation. & W & *\\ \hline

    $G[i][j]$ & $N_{sv,max} \times 3 \times 32$ & The $j$-th 32-bit floating-point element of the $i$-th SV's LOS vector. & - & R\\ \hline

    $C[i]$ & $N_{sv,max} \times 2$ & The saved constellation index (defined above) for the $i$-th SV. Translates to the index of the one hot column in the geometry matrix (i.e. $G[i][4+C[i]]=1$). & - & RW\\ \hline

    $N_{sv}$ & $\text{ceil} ( \text{log}_2 ( \newline N_{sv,max} ) )$ & The current number of SVs in view. Increments with each new written SV. & - & RW\\ \hline

    $N_{const}$ & $2$ & The current number of constellations with visible SVs. Taken as the maximum value of $C[i]$. & - & RW\\ \hline

    $N_{ss}$ & $\text{ceil} ( \text{log}_2 ( \newline N_{ss,max} ) )$ & The current number of formed subsets. Increments with each new calculated least-squares matrix. & - & RW\\ \hline

    $\sigma_{tropo}^2$ & $32$ & Variance for the tropospheric delay for the current SV. & W & R\\ \hline

    $\sigma_{user}^2$ & $32$ & Variance for the user delay for the current SV. & W & R\\ \hline

    $\sigma_{URA}^2$ & $32$ & Variance for the clock and ephemeris error used for integrity for the current SV. & W & R\\ \hline

    $\sigma_{URE}^2$ & $32$ & Variance for the clock and ephemeris error used for accuracy and continuity for the current SV. & W & R\\ \hline

    $C_{int}[i][i]$ & $N_{sv,max} \times 32$ & The $i$-th element of the diagonal of the covariance matrix for integrity. & - & RW\\ \hline

    $C_{acc}[i][i]$ & $N_{sv,max} \times 32$ & The $i$-th elemen tof the diagonal of the covariance matrix for accuracy and continuity. & - & RW\\ \hline

    $W^{1/2}[i][i]$ & $N_{sv,max} \times 32$ & Square root of the $i$-th element of the diagonal of the weight matrix. & - & RW\\ \hline

    $U[i][j]$ & $N_{sv,max} \times 7 \times 32$ & $i,j$-th element of the matrix resulting from $W^{1/2}G$. & - & RW\\ \hline

    $b_{nom}[i]$ & $N_{sv,max} \times 32$ & Maximum nominal bias for integrity for the $i$-th SV. & W & RW\\ \hline

    $idx_{ss}$ & $N_{ss,max}$ & Activation string to indicate the SVs that the current subset includes. & W & *\\ \hline

    $idx_{ss}[k]$ & $N_{ss,max} \times N_{ss,max}$ & Activation string for the $k$-th subset. & - & R\\ hline

    $y[i]$ & $N_{sv,max} \times 32$ & The $i$-th element of the vector to multiply with the least-squares matrix. & W & R\\ \hline

    $S[k][j][i]$ & $N_{ss,max} \times 7 \times N_{sv,max} \times 32$ & The $(j,i)$ element of the least-squares matrix for the $k$-th subset. & - & RW\\ \hline

    $spr[j][i]$ & $7 \times N_{sv,max} \times 32$ & The $(j,i)$-th element of the scratchpad matrix. & - & RW\\ \hline

    $K_{fa}[j]$ & $3 \times 32$ & The threshold factor for the global test of the $j$-th element of the coordinate vector (ENU). & W & R\\ \hline

    $K_{fa,r}$ & $32$ & The threshold factor for the local exclusion test. & W & R\\ \hline
  \end{tabular}
\end{center}
\end{table}

\section{RPU Exceptions}

\begin{table}[htbp] \begin{center}
  \caption{RPU Exceptions}\label{tab:exc}
  \begin{tabular}{|m{1.5cm}|m{3.5cm}|m{7cm}|}
    \hline
    \textbf{Code} & \textbf{Exception Name} & \textbf{Exception Description}\\ \hline

    $0$ & None & No exception, OKAY functionality.\\ \hline

    $1$ & Invalid instruction & The instruction opcode is invalid.\\ \hline

    $2$ & No implementation & The instruction is not implemented in the RPU.\\ \hline

    $4$ & Co-processor unusable & The RPU is currently unusable.\\ \hline

    $8$ & Trap & The executed instruction resulted in a trap.\\ \hline
  \end{tabular}
\end{center}
\end{table}

\end{document}

